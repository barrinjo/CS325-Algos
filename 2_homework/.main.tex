\documentclass{article}
\usepackage[utf8]{inputenc}
\usepackage{amsmath}
\usepackage{amssymb}
\usepackage{graphicx}

\title{CS325 Homework 2}
\author{Joshua Barringer \\ Prof. Schutfort}

\date{24 January 2020}

\begin{document}

\maketitle

\section*{Problem 1}

\subsection*{a) $T(n) = T(n - 2) + n$}

Solved by the substitution method:

\[
    T(n)
    \begin{cases} 
      1 & n = 0 \\
      T(n) = T(n - 2) + n & n > 0
   \end{cases}
\]

$T(n) = T(n - 2) + n$

$T(n) = [T(n - 4) + n] + n$

$T(n) = T(n - 4) + 2n$

$T(n) = [T(n - 6) + n] + 2n$

$T(n) = T(n - 6) + 3n$

$\;\;\;\vdots$

$T(n) = T(n - 2k) + kn$

Set $n - 2k$ to be equal to zero, which is the base case:

$n - 2k = 0$

$n = 2k$

$k = \frac{1}{2}n$

Plug result back into the function:

$T(n) = T(n - 2(\frac{1}{2}n)) + (\frac{1}{2}n)n$

$T(n) = T(0) + \frac{1}{2}n^2$

$T(n) = 1 + \frac{1}{2}n^2$

$\therefore$  \boxed{T(n) = \Theta(n^2)}

\subsection*{b) $T(n) = 4T(\frac{n}{2}) + n^3$}

Solved using master method:

$T(n) = aT(n/b) + f(n)$

$\therefore a = 4, b = 2$ and $f(n) = n^3$

\subsubsection*{1. If $f(n) = O(n^{log_ba - \epsilon})$ for some constant $\epsilon \geq  0$, then $T(n) = \Theta(n^{log_ba})$.}

$\vdots$

$log_ba = log_24 = 2$

$n^3 \neq O(n^{2 - \epsilon})$

\subsubsection*{2. If $f(n) = \Theta(n^{log_ba})$, then $T(n) = \Theta(n^{log_ba}lgn)$.}

$\vdots$

$n^3 \neq \Theta n^2$

\subsubsection*{3. If $f(n) = \Omega(n^{log_ba + \epsilon})$ for some constant $\epsilon > 0$, and if $af(n/b) \leq cf(n)$ for some constant $c < 1$ and all sufficiently large $n$, then $T(n) = \Theta(f(n))$.}

$\vdots$

$n^3 = \Omega(n^{2 + \epsilon})$

$\therefore$ \boxed{T(n) = \Theta(n^3)} due to the third rule of the Master Method.

\subsection*{c) $T(n) = 9T(\frac{n}{3}) + n^2$}

Solved using master method:

$T(n) = aT(n/b) + f(n)$

$\therefore a = 9, b = 3$ and $f(n) = n^2$

\subsubsection*{1. If $f(n) = O(n^{log_ba - \epsilon})$ for some constant $\epsilon \geq  0$, then $T(n) = \Theta(n^{log_ba})$.}

$\vdots$

$log_ba = log_39 = 2$

$n^3 \neq O(n^{2 - \epsilon})$

\subsubsection*{2. If $f(n) = \Theta(n^{log_ba})$, then $T(n) = \Theta(n^{log_ba}lgn)$.}

$\vdots$

$n^3 = \Theta n^2$

$\therefore$ \boxed{T(n) = \Theta(n^2lgn)} due to the second rule of the Master Method.

\section*{Problem 2}

\[T(n) = 4T(\frac{n}{3}) + n + 1\]

$T(n) = aT(n/b) + f(n)$

$\therefore a = 4, b = 3$ and $f(n) = n + 1$

$log_34 = 1.2619$

$n + 1 = O(n^{1.2619 - \epsilon})$

$\therefore$ \boxed{T(n) = \Theta(n^{log_34}) = \Theta(n^{1.2619})}

\section*{Problem 3}

\subsection*{a)}

\begin{verbatim}

    search(left, right) {
        if(right > 1) {
        size = right - left
        mid_left = size/3 + l
        mid_right = size/3 + 2l
            if(data[mid_left] == target) {
                return mid_left
            }
            if(data[mid_right] == target) {
                return mid_right
            }
            else {
                search(left, mid_left)
                search(mid_left, mid_right)
                search(mid_right, right)
            }
        } else {
            return n
        }
    }
    
\end{verbatim}

\subsection*{b) \boxed{T(n) = 3T(n/3) + 1}}

\subsection*{c) Solved using master method:}

$T(n) = aT(n/b) + f(n)$

$\therefore a = 3, b = 3$ and $f(n) = 1$

\subsubsection*{1. If $f(n) = O(n^{log_ba - \epsilon})$ for some constant $\epsilon \geq  0$, then $T(n) = \Theta(n^{log_ba})$.}

$\vdots$

$log_ba = log_33 = 1$

$1 = O(n^{1 - \epsilon})$

$\therefore$ \boxed{T(n) = \Theta(n)} solved with the first rule of the master method.

\section*{Problem 4}

\subsection*{a)}

\begin{verbatim}
    
    Merge3(array) {
        if(array.length > 2) {
            left_mid = array.length//3
            right_mid = 2*(array.length//3)
            
            left_array = array[:left_mid]
            center_array = array[left_mid:right_mid]
            right_array = array[right_mid:]
            
            merge3(left_array)
            merge3(center_array)
            merge3(right_array)
            
            merge(left_array, center_array, right_array)
        } else if(array.length == 2) {
            if(array[0] > array[1]) 
                swap(array[0], array[1])
        }
    }
    
\end{verbatim}

\subsection*{b) \boxed{T(n) = 3T(n/3) + n}}

\subsection*{c) Solved with the master method:}

$T(n) = aT(n/b) + f(n)$

$\therefore a = 3, b = 3$ and $f(n) = n$

\subsubsection*{1. If $f(n) = O(n^{log_ba - \epsilon})$ for some constant $\epsilon \geq  0$, then $T(n) = \Theta(n^{log_ba})$.}

$\vdots$

$log_ba = log_33 = 1$

$n \neq O(n^{1 - \epsilon})$

\subsubsection*{2. If $f(n) = \Theta(n^{log_ba})$, then $T(n) = \Theta(n^{log_ba}lgn)$.}

$\vdots$

$n = \Theta n^1$

$\therefore$ \boxed{T(n) = \Theta(nlgn)} due to the second rule of the Master Method.

\section*{Problem 5}

\subsection*{c)}

    \begin{center}
        \begin{tabular}{||c c||} 
        \hline
        n & time(s) \\ [0.5ex] 
        \hline\hline
        1000 & 0.00187587738 \\
        \hline
        2000 & 0.003946065903 \\
        \hline
        3000 & 0.006802558899 \\
        \hline
        4000 & 0.009682178497 \\
        \hline
        5000 & 0.01127719879 \\
        \hline
        6000 & 0.01403903961 \\
        \hline
        7000 & 0.01804995537 \\
        \hline
        8000 & 0.02071690559 \\
        \hline
        9000 & 0.02333903313 \\
        \hline
        10000 & 0.02658748627 \\
        \hline
        11000 & 0.03029489517 \\
        \hline
        12000 & 0.03291273117 \\
        \hline
        13000 & 0.03332877159 \\
        \hline
        14000 & 0.03526353836 \\
        \hline
        15000 & 0.03863358498 \\
        \hline
        \end{tabular}
    \end{center}
    
\subsection*{d)}

\includegraphics[width=\textwidth]{oneg.png}

A linear curve, $y = mx+b$, fits the data best.
The equation that best fits the data is \boxed{y = 0.00000270786x-0.0012129}

\subsection*{e)}

\includegraphics[width = \textwidth]{twog.png}

The merge3Time runs slightly faster than the mergeTime, but their curves are both linear, so they're significantly faster than insertion sort.  Mergesort and Mergesort3 have the same theoretical times.  The times were collected on the same computer, so the difference in times are likely due to the code I wrote.  The difference in times are small enough to be insignificant, so the times make sense based on their theoretical running times.

\end{document}
