\documentclass{article}
\usepackage[utf8]{inputenc}
\usepackage{amsmath}
\usepackage{amssymb}

\title{CS325 Homework 2}
\author{Joshua Barringer}
\date{24 January 2020}

\begin{document}

\maketitle

\section*{Problem 1}

\subsection*{a) $T(n) = T(n - 2) + n$}

Solved by the substitution method:

\[
    T(n)
    \begin{cases} 
      1 & n = 0 \\
      T(n) = T(n - 2) + n & n > 0
   \end{cases}
\]

$T(n) = T(n - 2) + n$

$T(n) = [T(n - 4) + n] + n$

$T(n) = T(n - 4) + 2n$

$T(n) = [T(n - 6) + n] + 2n$

$T(n) = T(n - 6) + 3n$

$\;\;\;\vdots$

$T(n) = T(n - 2k) + kn$

Set $n - 2k$ to be equal to zero, which is the base case:

$n - 2k = 0$

$n = 2k$

$k = \frac{1}{2}n$

Plug result back into the function:

$T(n) = T(n - 2(\frac{1}{2}n)) + (\frac{1}{2}n)n$

$T(n) = T(0) + \frac{1}{2}n^2$

$T(n) = 1 + \frac{1}{2}n^2$

$\therefore$  \boxed{T(n) = \Theta(n^2)}

\subsection*{b) $T(n) = 4T(\frac{n}{2}) + n^3$}

Solved using master theorem:

$T(n) = aT(n/b) + f(n)$

$\therefore a = 4, b = 2$ and $f(n) = n^3$

\subsubsection*{1. If $f(n) = O(n^{log_ba - \epsilon})$ for some constant $\epsilon \geq  0$, then $T(n) = \Theta(n^{log_ba})$.}

$\vdots$

$log_ba = log_24 = 2$

$n^3 \neq O(n^{2 - \epsilon})$

\subsubsection*{2. If $f(n) = \Theta(n^{log_ba})$, then $T(n) = \Theta(n^{log_ba}lgn)$.}

$\vdots$

$n^3 \neq \Theta n^2$

\subsubsection*{3. If $f(n) = \Omega(n^{log_ba + \epsilon})$ for some constant $\epsilon > 0$, and if $af(n/b) \leq cf(n)$ for some constant $c < 1$ and all sufficiently large $n$, then $T(n) = \Theta(f(n))$.}

$\vdots$

$n^3 = \Omega(n^{2 + \epsilon})$

$\therefore$ \boxed{T(n) = \Theta(n^3)} due to the third rule of the Master Theorem.

\subsection*{c) $T(n) = 9T(\frac{n}{3}) + n^2$}

Solved using master theorem:

$T(n) = aT(n/b) + f(n)$

$\therefore a = 9, b = 3$ and $f(n) = n^2$

\subsubsection*{1. If $f(n) = O(n^{log_ba - \epsilon})$ for some constant $\epsilon \geq  0$, then $T(n) = \Theta(n^{log_ba})$.}

$\vdots$

$log_ba = log_39 = 2$

$n^3 \neq O(n^{2 - \epsilon})$

\subsubsection*{2. If $f(n) = \Theta(n^{log_ba})$, then $T(n) = \Theta(n^{log_ba}lgn)$.}

$\vdots$

$n^3 = \Theta n^2$

$\therefore$ \boxed{T(n) = \Theta(n^2lgn)} due to the second rule of the Master Theorem.

\section*{Problem 2}

\[T(n) = 4T(\frac{n}{3}) + n + 1\]

$T(n) = aT(n/b) + f(n)$

$\therefore a = 4, b = 3$ and $f(n) = n + 1$

$log_34 = 1.2619$

$n + 1 = O(n^{1.2619 - \epsilon})$

$\therefore$ \boxed{T(n) = \Theta(n^{log_34}) = \Theta(n^{1.2619})}

\section*{Problem 3}

\subsection*{a)}

\begin{verbatim}

    search(left, right) {
        if(right > 1) {
        size = right - left
        mid_left = size/3 + l
        mid_right = size/3 + 2l
            if(data[mid_left] == target) {
                return mid_left
            }
            if(data[mid_right] == target) {
                return mid_right
            }
            else {
                search(left, mid_left)
                search(mid_left, mid_right)
                search(mid_right, right)
            }
        } else {
            return n
        }
    }
    
\end{verbatim}

\subsection*{b) $T(n) = 3T(n/3) + 2$}

Solved using master theorem:

$T(n) = aT(n/b) + f(n)$

$\therefore a = 3, b = 3$ and $f(n) = 2$

\subsubsection*{1. If $f(n) = O(n^{log_ba - \epsilon})$ for some constant $\epsilon \geq  0$, then $T(n) = \Theta(n^{log_ba})$.}

$\vdots$

$log_ba = log_33 = 1$

$2 = O(n^{1 - \epsilon})$

\therefore \boxed{T(n) = \Theta(n)}

\subsection*{c)}

\section*{Problem 4}

\end{document}
