\documentclass{article}
\usepackage[utf8]{inputenc}
\usepackage{amsmath}
\usepackage{amssymb}
\usepackage{graphicx}

\title{CS325 Homework 3}
\author{Josuha Barringer \\ Prof. Schutfort}
\date{7 February 2020}

\begin{document}

\maketitle

\section*{Problem 1: Rod Cutting}

We define a rod with length 3, whose length prices are as follows, [1, 8, 10].  Using the greedy algorithm, the value per inch for each length would be [$\frac{1}{1}$, $\frac{8}{2}$, $\frac{10}{3}$], or [1, 4, 3.$\overline{333}$].  The greedy algorithm would first cut off a rod of length 2, since it has the highest value per inch of 4.  The remaining 1 inch cannot be cut any more, so the algorithm would finish with a total value of 9.  The actual highest possible value would be to use no cuts and end with a total value of 10.  Clearly, the "greedy" strategy doesn't work in this situation.

\section*{Problem 2: Modified Rod Cutting}

I modified the book's Bottom-Up-Cut-Rod Method for this problem:\\

\textsc{Bottom-Up-Cut-Rod$(p,n,cut\_price)$}

    \begin{verbatim}
        1 let r[0..n] be a new array
        2 r[0] = 0
        3 for j = 1 to n
        4     q = -infinity
        5     for i = 1 to j
        6         q = max(q, p[i] + r[j - i] - cut_price)
        7     r[j] - q
        8 return r[n]
    \end{verbatim}
    
This modified algorithm takes a third argument, "cut\_price".  On line 6, the previous max variable q is now compared with the proposed new cut summed with the memo lookup for length optimization for the remaining piece, \textit{with the price of the cut subtracted from the total}.  This modification should allow custom cut prices, as well as account for the cut price during the rod cutting optimization.

\section*{Problem 3}

\section*{Problem 4}

\end{document}
